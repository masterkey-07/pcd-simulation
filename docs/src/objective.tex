\section{Objetivo}

O objetivo deste estudo é desenvolver uma simulação eficiente para modelar a difusão de contaminantes em corpos d'água, como lagos e rios, utilizando técnicas avançadas de paralelismo e otimização de código. A simulação visa não apenas prever o comportamento dos poluentes ao longo do tempo, mas também avaliar o desempenho computacional de diferentes abordagens de paralelização.

As principais metas incluem:

\begin{itemize}
    \item Implementar e comparar técnicas de paralelismo utilizando OpenMP, CUDA e MPI para acelerar a simulação.
    \item Avaliar o impacto da otimização -O3 do GCC no desempenho da simulação.
    \item Analisar a eficiência e a escalabilidade de cada técnica em diferentes cenários de simulação, variando o tamanho da matriz e o número de iterações.
    \item Fornecer insights sobre a melhor abordagem para otimizar a execução de simulações de difusão de contaminantes em ambientes computacionais diversos.
\end{itemize}

Ao alcançar esses objetivos, o estudo pretende contribuir para a compreensão dos benefícios e limitações das diferentes técnicas de paralelismo e otimização, assim destacando qual método demonstrou a melhor performance para o problema proposto.
