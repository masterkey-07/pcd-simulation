\section{Descrição do Problema} \label{sec:firstpage}

O objetivo deste estudo é otimizar uma simulação de difusão de contaminantes em corpos d'água, como lagos e rios, utilizando técnicas avançadas de paralelismo e otimização de código. A simulação visa não apenas prever o comportamento dos poluentes ao longo do tempo, mas também avaliar o desempenho computacional de diferentes abordagens de paralelização.

A função que descreve a simulação está detalhada logo abaixo:

\[\frac{\partial C}{\partial t}\ = D \cdot (\nabla)^{2} C \]

Onde:
\begin{itemize}
    \item \( C \) é a concentração do contaminante,
    \item \( t \) é o tempo,
    \item \( D \) é o coeficiente de difusão, e
    \item \( (\nabla)^{2} C \) representa a taxa de variação de concentração no espaço.
\end{itemize}

Este modelo descreve a difusão de um contaminante em um meio homogêneo e isotrópico. A equação baseia-se na Segunda Lei de Fick, que afirma que a taxa de variação temporal da concentração em um ponto é proporcional à divergência do fluxo de difusão nesse ponto.

Para resolver esta equação, utiliza-se o método de diferenças finitas, discretizando o espaço e o tempo. A discretização leva à equação de diferença:

\[
    \frac{C^{n+1}_{i,j} - C^{n}_{i,j}}{\Delta t} = D \cdot \left( \frac{C^{n}_{i+1,j} + C^{n}_{i-1,j} + C^{n}_{i,j+1} + C^{n}_{i,j-1} - 4C^{n}_{i,j}}{\Delta x^2} \right)
\]

Onde:
\begin{itemize}
    \item \( C^{n}_{i,j} \) é a concentração no ponto \( (i,j) \) no tempo \( n \),
    \item \( \Delta t \) é o passo temporal, e
    \item \( \Delta x \) é o tamanho da célula no espaço discreto.
\end{itemize}

Este esquema permite calcular a evolução temporal da concentração em uma grade 2D, sendo adequado para implementação em sistemas paralelos, como o modelo estudado neste trabalho.
