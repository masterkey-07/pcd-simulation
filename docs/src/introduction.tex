\section{Introdução}

A difusão de contaminantes em corpos d'água, como lagos, rios e oceanos, é um tema de grande relevância ambiental e social, dada a crescente preocupação com a qualidade da água e o impacto de atividades humanas no meio ambiente. A contaminação por substâncias químicas, resíduos industriais, efluentes domésticos e outros poluentes pode comprometer ecossistemas aquáticos e a saúde pública. Assim, entender e prever a propagação desses contaminantes é fundamental para a formulação de políticas de controle e mitigação.

A partir de simulações computacionais pode ser feito o estudo no comportamento de difusão de poluentes na água. As simulações permitem a análise de diferentes cenários e variáveis, oferecendo uma visão detalhada sobre o comportamento dos poluentes ao longo do tempo e em diferentes condições ambientais. Essas ferramentas proporcionam aos pesquisadores a capacidade de modelar e prever eventos de contaminação.

No entanto, pode ser alta o custo computacional ao simular processos de difusão em grande escala. Para superar esses desafios, métodos de otimização do algoritmo são necessárias para agilizar o tempo de execução das simulações, como também ser capaz de computar casos mais demandantes. Tecnologias como OpenMP (Open Multi-Processing), CUDA (Compute Unified Device Architecture) e MPI (Message Passing Interface) oferecem soluções poderosas para acelerar os cálculos, permitindo o processamento simultâneo de múltiplas operações e a distribuição de tarefas entre diferentes unidades de processamento.

Este estudo se concentra na implementação e comparação dessas técnicas de otimização em simulações de difusão de contaminantes, avaliando seu impacto no desempenho e na precisão dos modelos. Através da utilização de OpenMP, CUDA e MPI, será buscado a redução do tempo de execução de um algoritmo existente de simulação do processo de difusão em corpos d'água.
