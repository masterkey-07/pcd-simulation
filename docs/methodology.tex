\section{Metodologia}

Para otimizar a simulação da difusão de contaminantes em corpos d'água, diversas técnicas de paralelismo e otimização de código foram empregadas. Nesta seção, são descritas as metodologias relacionadas ao uso de CUDA, OpenMP, MPI e a otimização -O3 do GCC, com o objetivo de melhorar o desempenho computacional da simulação.

\subsection{CUDA (Compute Unified Device Architecture)}

CUDA é uma plataforma de computação paralela desenvolvida pela NVIDIA, que permite o uso de GPUs (Graphics Processing Units) para realizar cálculos computacionais intensivos. Ao distribuir as operações de cálculo em milhares de núcleos da GPU, CUDA possibilita uma aceleração significativa em tarefas que envolvem grandes volumes de dados ou computações repetitivas, como na simulação de difusão de contaminantes. A implementação em CUDA envolve a escrita de kernels, que são funções executadas em paralelo pela GPU, e a gestão eficiente da memória entre a CPU e a GPU.

\subsection{OpenMP (Open Multi-Processing)}

OpenMP é uma API que suporta a programação paralela em ambientes compartilhados de memória. Com uma sintaxe baseada em diretivas, OpenMP facilita a paralelização de loops e outras estruturas de controle em linguagens como C, C++ e Fortran. A utilização de OpenMP nesta simulação permite a execução paralela em múltiplos threads, aproveitando melhor os recursos de CPUs multicore. A abordagem é especialmente útil para dividir o trabalho computacional entre os núcleos do processador, reduzindo o tempo total de execução.

\subsection{MPI (Message Passing Interface)}

MPI é uma biblioteca padrão para programação paralela em sistemas de memória distribuída. Ao contrário do OpenMP, que opera em memória compartilhada, MPI permite a comunicação entre diferentes processos que podem estar executando em máquinas distintas. Na simulação de difusão de contaminantes, o MPI é utilizado para distribuir partes do domínio de cálculo entre múltiplos nós de um cluster de computadores, sincronizando e agregando os resultados. Isso permite o escalonamento eficiente da simulação para conjuntos de dados muito grandes.

\subsection{Otimização -O3 do GCC}

A flag -O3 do compilador GCC (GNU Compiler Collection) ativa uma série de otimizações agressivas de código que visam maximizar o desempenho de programas compilados. Entre as otimizações realizadas estão a unrolling de loops, a substituição de expressões comuns, a eliminação de código redundante, e a inlining de funções. Essas otimizações são aplicadas automaticamente pelo GCC durante a compilação do código, reduzindo o tempo de execução e melhorando a eficiência do programa sem a necessidade de modificações manuais no código-fonte.
