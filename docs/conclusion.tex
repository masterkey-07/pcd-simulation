\section{Conclusão}

Os resultados obtidos evidenciam a importância das técnicas de paralelização para a redução do tempo de execução em simulações computacionais intensivas. A implementação sequencial não otimizada apresentou um tempo de execução significativamente superior às demais abordagens, demonstrando a ineficiência de métodos não paralelos para problemas de grande escala.

A versão sequencial otimizada reduziu o tempo de execução, mas ainda permaneceu consideravelmente mais lenta do que as abordagens paralelas. Entre as técnicas de paralelização analisadas, a implementação via CUDA apresentou o melhor desempenho, alcançando uma redução expressiva no tempo de execução em comparação com todas as outras abordagens. O uso de OpenMP e MPI também demonstrou ganhos significativos de desempenho, mas ainda superiores ao tempo registrado pela abordagem CUDA.

Portanto, os resultados indicam que a utilização de GPUs com CUDA se mostra particularmente eficiente para problemas do tipo stencil 2D, onde o processamento de células pode ser altamente paralelizado devido à sua natureza estruturada e ao alto grau de reutilização de dados vizinhos. Essa característica permite um melhor aproveitamento da hierarquia de memória da GPU, reduzindo a latência e aumentando a largura de banda efetiva. Trabalhos futuros podem explorar diferentes estratégias de acesso à memória global e compartilhada para melhorar ainda mais a eficiência da computação stencil em arquiteturas CUDA.